Práce pro mě byla velice přínosná z hlediska osobních zkušeností a získání nových dovedností. Certifikace kterou jsem získal si velice vážím a věřím, že se mi bude v praxi hodit. Konference Cassandra Summit, na které prezentovali inženýři z předních technických firem využívajících Cassandru byla obrovským přínosem pro tuto práci a mě samotného. V budoucnu se opět rád takovéto akce zúčastním. 

\subsection{Roztříštěnost informací}
Během prvotního sběru informací jsem se potýkal s problémy, že informace ohledně trendu BigData jsou velice rotříštěné a existuje mnoho výkladů a také se články většinou zaměřovali pouze na konkrétní pododvětví a nikde nebyly žádné ucelené informace. Vzhledem k rozsáhlému prostudování problematiky, věřím, že se mi podařilo v prvních kapitolách stručně a rozumně shrnout problematiku BigData včetně základních konceptů a historie. 

\subsection{Nevyspělost technologií}
Pravděpodobně největším problémem se kterým jsem se během práce setkal, byla značná nevyspělost technologií. Během popisování technologií a softwaru v Apache BigData stacku, které přinášejí komplexní funkce a tvaří se jako plně funkční platforma pro práci s BigData není situace úplně růžová. Tyto softwarové nástroje se rapidně vyvýjí a mají mnoho verzí, které jsou dost často zpětně nekompatibilní a co hůř určité verze nejsou kompatibilní s ostatním softwarem z této platformy.  Což je klíčový problém při navrhování architektury systémů, které využívají 2 a více těchto nástrojů pohromadě, když zjistíme, že musíme oželet nějakou novou funkci, protože nová verze není kompatibilní s ostatním softwarem. Což tvoří z výběru vhodných verzi téměř alchemickou činost. Tento problém považuji obecně za nejvetší překážku při využívání a implementaci této platformy. Z internetových diskusí a také na Cassandra summitu se tomuto tématu vývojáři hodně věnují a snaží se sdílet své zkušenosti pro ostatní.  

\subsection{Příprava případů užití pro předmět na ČVUT}
Nevyspělost technologií byla i velkou překážkou při tvorbě vlastních případů užití. Tento problém částečně vyřešila aktualizace souhrného balíku firmy Datastax, který situaci zlepšil. Zvolené případy užittí jsou přiložené na CD a jsou připraveny k využití během teoretické i praktické výuky v nově připravovaném předmětu. S vedoucím práce jsme kladli velký důraz na široky záběr použitých technologií a praktičnost těchto příkladů. Zvolené příklady toto kritérium splňují a zároveň se zaměřují hodně do hloubky na databázi Cassandra, která byla zvolena, jako hlavní náplň této práce, jakožto v současnosti nejpoužívanější průmyslové řešení. Kromě praktických příkladů se práce může využít k tvorbě teoretických podkladů pro výuku tohoto předmětu a to převážně části věnující se popisu architektury Cassandry a úvodní kapitola shrnující koncept BigData a historii tohoto konceptu. 