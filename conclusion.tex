Tato práce pro mě byla velice přínosná z hlediska získání nových zkušeností a dovedností. Certifikace, kterou jsem získal, si velice vážím a věřím, že se mi bude v praxi hodit. Konference Cassandra Summit, na které prezentovali inženýři z předních technických firem využívajících Cassandru, byla obrovským přínosem pro tuto práci i mě samotného. V budoucnu se opět rád takovéto akce zúčastním. 

\subsection{Roztříštěnost informací}
Během prvotního sběru informací jsem se potýkal s problémy, že informace ohledně BigData jsou velice roztříštěné a existuje mnoho různých výkladů. Většina článků se zaměřovala pouze na konkrétní pododvětví a nikde nebyly žádné ucelené informace. Vzhledem k rozsáhlému prostudování problematiky věřím, že se mi podařilo v prvních kapitolách stručně a rozumně shrnout problematiku BigData včetně základních konceptů a historie. 

\subsection{Nevyspělost technologií}
Pravděpodobně největším problémem, se kterým jsem se během tvorby práce setkal, byla značná nevyspělost technologií. Během popisování technologií a softwaru v Apache BigData Stacku, který přináší komplexní funkce a tváří se jako plně funkční platforma pro práci s BigData, jsem zjistil, že situace není tak úplně růžová. Tyto softwarové nástroje se rapidně vyvíjí a mají mnoho verzí, které jsou dost často zpětně nekompatibilní, a co hůř, určité verze nejsou kompatibilní s ostatním softwarem z této platformy. To je ovšem závažný problém při navrhování architektury systémů, které využívají více těchto nástrojů pohromadě. Může se snadno stát, že zjistíme, že musíme oželet nějakou novou funkci, protože nová verze není kompatibilní s ostatním softwarem. Z výběru vhodných verzí tak máme téměř alchemickou činnost. Tento problém považuji obecně za největší překážku při využívání a implementaci této platformy. Na internetových diskusích a také na Cassandra summitu se tomuto tématu vývojáři hodně věnovali a snažili se sdílet své zkušenosti s ostatními.  

\subsection{Příprava případů užití pro předmět na ČVUT}
Nevyspělost technologií byla velkou překážkou také při tvorbě vlastních případů užití. Tento problém částečně vyřešila aktualizace souhrnného balíku firmy Datastax, který situaci zlepšil. Zvolené případy užití jsou přiložené na CD a jsou připraveny k využití během teoretické i praktické výuky v nově připravovaném předmětu. S vedoucím práce jsme kladli velký důraz na široký záběr použitých technologií a praktičnost těchto příkladů. Zvolené příklady toto kritérium splňují a zároveň se do hloubky zaměřují na databázi Cassandra, jakožto v současnosti nejpoužívanější průmyslové řešení. Kromě praktických příkladů je možné práci využít k tvorbě teoretických podkladů pro výuku tohoto předmětu a to převážně části věnující se popisu architektury Cassandry a úvodní kapitola shrnující koncept BigData a jeho historii. 