Pro výběr tématu mé závěrečné práce mě motivovalo hned několik věcí. BigData jsou moderním trendem a technologií, která v současné době hýbe IT světem. Na druhou stranu je nutno říci, že je tento pojem značně zprofanovaný jeho až příliš častým využitím v marketingu konkrétních produktů. Mou první motivací tak bylo přinést ucelený a objektivní pohled na to, co to BigData jsou a naopak jasně vyčlenit, co nejsou tak, aby si čtenář dokázal po přečtení práce pod tímto pojmem jasně představit konkrétní technologie, metody a možnosti využití. 

Jak již bylo zmíněno, BigData jsou ve světě poměrně novým trendem. Další motivací tak pro mě bylo naučit se s něčím novým a zajímavým, s něčím s čím se mnoho lidí ještě nezabývá. Po oslovení vedoucího práce a po společné debatě nad možností využití tohoto tématu jako mé závěrečné práce, jsme došli k několika jasným cílům, které by tato práce měla naplnit. 

\subsection{Zmapování situace kolem BigData}
Jak již bylo zmíněno, jedním z cílů je poodhalit problematiku BigData jako takovou, k čemuž je nutné provést zmapování současného stavu v této oblasti. Zjistit nejčastější případy využití v akademické sféře nebo průmyslu, zmapovat aktuálně používané technologie a zároveň prozkoumat všechny přidružené technologie. 

\subsection{Využití zkušeností a znalostí a jejich rozvoj}
S NoSQL databázemi jsem měl předchozí zkušenosti, které jsem chtěl zužitkovat a prostřednictvím této práce navíc posunout mé znalosti dál. V rámci přípravy této práce jsem se zúčastnil Cassandra Summitu 2013 v Londýně, kde jsem rovněž absolvoval Cassandra Developer Training a získal jsem oficiální certifikát na vývoj s databází Cassandra přímo od jejího výrobce firmy Datastax. Vzhledem k tomu, že se práce zabývá především Cassandrou, považuji toto školení za velký úspěch a samotný summit za velice přínosný především pro kapitolu věnovanou případům využití.

\subsection{Příprava nového předmětu na ČVUT}
BigData jsou moderní technologickou i vědní disciplínou a Fakulta informačních technologií (konkrétně Katedra softwarového inženýrství) projevila zájem o vytvoření nového předmětu, který by se po praktické stránce věnoval BigData a databázovým systémům. Pod dohledem vedoucího katedry a vedoucího práce Ing. Michala Valenty, PhD tento předmět připravuje Ing. Josef Gattermayer a tato práce by měla sloužit jako pomůcka pro přípravu tohoto předmětu a to především po stránce praktických příkladů, které z ní vzejdou a poslouží jako náměty na možná témata cvičení, či rovnou jako hotová řešení. 

\subsection{}
V následujících kapitolách se postupně věnuji všem třem výše vytyčeným cílům, které se prolínají se zadáním této práce. 

