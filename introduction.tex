Pro výběr tohoto tématu mé závěrečné práce mě motivovalo hned několik věcí. BigData jsou moderním trendem a technologií, která v současné době hýbe IT světem. Nadruhou stranu je nutno říci, že je to také velmi často využívaný termín převážně v marketingových kruzích. Mou první motivací bylo přínést ucelený objektivní pohled na to, co to jsou BigData a naopak jasně vyčlenit co nejsou tak, aby si čtenář dokázal po přečtení práce pod tímto pojmem jasně představit konkrétní technologie, metody a možnosti využití. 

Jak bylo zmíněno BigData jsou novým trendem a další motivací pro mě bylo naučit se s nečím novým, zajímavým s něčím s čím se mnoho lidí nezabývá. Po oslovení vedoucího práce a debatou nad možností tohoto tématu jako mé závěrečné práce jsme došli k několika jasným cílům této práce. 

\subsection{Zmapování situace kolem BigData}
Jak již bylo zmíněno jedním z cílů je poodhalit problematiku BigData a k tomu zároveň patří zmapování současného stavu kolem tohoto tématu. Zjistit nejčastější případy využití v akademické sféře nebo průmyslu. Zmapovat aktuálně používané technologie a zaroveň prozkoumat všechny přidružené technologie. 

\subsection{Využít zkušenosti a znalostí a jejích rozvoj}
S NoSQL databázemi jsem měl předchozí zkušenosti a chtěl jsem tyto zkušenosti zužitkovat, prostřednictvím této práce a navíc posunout mé znalosti dál. V příprav této přáce jsem zúčastnil Cassandra Summitu 2013 v Londýně, kde jsem rovněž absolvoval Cassandra Developer Training a získal jsem oficiální certifikát na vývoj s databází Cassandra přímo od jejího výrobce firmy Datastax. Vzhledem k tomu, že se práce zabývá především Cassandrou, považuji toto školení za velký úspěch a samotný summit za velice přínosný hlavně pro kapitolu věnovanou případům využití.

\subsection{Příprava nového předmětu na ČVUT}
BigData jsou moderní technologickou i vědní disciplínou a fakulta informačních technologií, konkrétně katedra softwarového inženýrství projevila zájem o vytvoření nového předmětu, který se po praktické stránce věnuje BigData a databázovým systémum. Pod dohledem vedoucího katedry a vedoucího této práce Ing. Michala Valenty, PhD tento předmět připravuje Ing. Josef Gattermayer a tato práce by měla sloužit jako pomůcka pro připravu tohoto předmětu a to především po stránce praktických příkladů, které vzejdou z této práce a poslouží jako náměty na možná témata cvičení, či rovnou jako hotová řešení. 

\subsection{}
V následujících kapitolách se postupně věnuji všem třem vytyčeným cílům výše, které se prolínají se zadáním této práce. 

