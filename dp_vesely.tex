% options:
% thesis=B bachelor's thesis
% thesis=M master's thesis
% czech thesis in Czech language
% slovak thesis in Slovak language
% english thesis in English language
% hidelinks remove colour boxes around hyperlinks

\documentclass[thesis=M,czech]{FITthesis}[2012/06/26]

\usepackage[utf8]{inputenc} % LaTeX source encoded as UTF-8

\usepackage{graphicx} %graphics files inclusion
% \usepackage{amsmath} %advanced maths
% \usepackage{amssymb} %additional math symbols
\usepackage{wrapfig}
\usepackage{tabularx}
%\usepackage{float}
%\restylefloat{table}

\usepackage{dirtree} %directory tree visualisation

\usepackage{listings}
\usepackage{color}

\definecolor{dkgreen}{rgb}{0,0.6,0}
\definecolor{gray}{rgb}{0.5,0.5,0.5}
\definecolor{mauve}{rgb}{0.58,0,0.82}

\lstset{frame=tb,
  language=SQL,
  aboveskip=3mm,
  belowskip=3mm,
  showstringspaces=false,
  columns=flexible,
  basicstyle={\small\ttfamily},
  numbers=none,
  numberstyle=\tiny\color{gray},
  keywordstyle=\color{blue},
  commentstyle=\color{dkgreen},
  stringstyle=\color{mauve},
  breaklines=true,
  breakatwhitespace=true
  tabsize=3
}

% % list of acronyms
% \usepackage[acronym,nonumberlist,toc,numberedsection=autolabel]{glossaries}
% \iflanguage{czech}{\renewcommand*{\acronymname}{Seznam pou{\v z}it{\' y}ch zkratek}}{}
% \makeglossaries

\newcommand{\tg}{\mathop{\mathrm{tg}}} %cesky tangens
\newcommand{\cotg}{\mathop{\mathrm{cotg}}} %cesky cotangens

% % % % % % % % % % % % % % % % % % % % % % % % % % % % % % 
% ODTUD DAL VSE ZMENTE
% % % % % % % % % % % % % % % % % % % % % % % % % % % % % % 

\department{Katedra \ldots (Softwarového inženýrství)}
\title{Distrubuované ukládání a zpracování velkého množství dat - případové studie}
\authorGN{Dominik} %(křestní) jméno (jména) autora
\authorFN{Veselý} %příjmení autora
\authorWithDegrees{Bc. Dominik Veselý} %jméno autora včetně současných akademických titulů
\supervisor{Ing. Michal Valenta, PhD}
\acknowledgements{Doplňte, máte-li komu a za co děkovat. V~opačném případě úplně odstraňte tento příkaz.}
\abstractCS{V~několika větách shrňte obsah a přínos této práce v~češtině. Po přečtení abstraktu by se čtenář měl mít čtenář dost informací pro rozhodnutí, zda chce Vaši práci číst.}
\abstractEN{Sem doplňte ekvivalent abstraktu Vaší práce v~angličtině.}
\placeForDeclarationOfAuthenticity{V~Praze}
\declarationOfAuthenticityOption{4} %volba Prohlášení (číslo 1-6)
\keywordsCS{Nahraďte seznamem klíčových slov v češtině oddělených čárkou.}
\keywordsEN{Nahraďte seznamem klíčových slov v angličtině oddělených čárkou.}

\begin{document}
	
% \newacronym{CVUT}{{\v C}VUT}{{\v C}esk{\' e} vysok{\' e} u{\v c}en{\' i} technick{\' e} v Praze}
% \newacronym{FIT}{FIT}{Fakulta informa{\v c}n{\' i}ch technologi{\' i}}

\begin{introduction}
	%sem napište úvod Vaší práce
\end{introduction}



\chapter{Cíl práce}
Zmínit že práce nikterak nesrovnává klasické a BigData databáze naopak se zabývá open source nástroji od apache


\section{Motivace}
\section{Příprava předmětu}
\section{Rozšíření povědomí}
\section{Praktické zkušenosti}


\chapter{Co jsou to BigData}



Popsat termín BigData není úplně snadné, a to hned z~několika důvodů. Především proto, že neexistuje žádná přesná definice tohoto pojmu. Tento termín je stejně jako obor, kterého se týká, velice dynamický a rychle se mění. K~obtížnější definici pojmu přispívá také fakt, že se hojně používá v~marketingové komunikaci jako tzv. \uv{buzzword} za účelem vzbudit zájem čtenáře/posluchače, přestože může být použit v~nesprávném kontextu.

Termín BigData označuje manipulaci s~datasety tak velkými, že je nemožné nebo velice obtížné s~nimi manipulovat za pomocí tradičních nástrojů a databází (převážně relačních). Pod pojmem manipulace s~datasety myslíme:

\begin{itemize}
  \item Sběr
  \item Organizace
  \item Ukládání
 \item Prohledávání
 \item Sdílení
 \item Analýza
 \item Vizualizace
\end{itemize}

Nalezení této hranice či její přesná definice je komplikovanější problém mimo rozsah této práce. Na toto téma již bylo publikováno mnoho jiných prací. Jak jsem zmínil již v~úvodu, práce se dále také nezabývá porovnáním BigData a klasických relačních databází. 

V~této kapitole se pokusím obecně přiblížit, co to tedy BigData jsou, jak se liší a důvod vzniku tohoto odvětví.


\section{Trend velkých datasetů}
Jak jsem již naznačil, BigData se věnují zpracování velkých datasetů. Trend upřednostňování velkých datasetů oproti několika menším, které v~součtu mají stejný objem a nesou stejné množství informace, začal vzhledem k~jednoduššímu hledání a objevení i zdánlivě neexistujících korelací, projevení obchodních trendů a vyhodnocování dat v~reálném nebo skoro reálném čase.

\section{3V}
Jak naznačil předchozí odstavec, BigData nejsou pouze o~objemu zpracovávaných dat, jak by se na první pohled mohlo zdát. Jedná se o~komplexnější kategorizaci, kde hrají roli i ostatní charakteristiky, které se v~literatuře značí zkratkou 3V, odvozenou od počátečních písmen těchto kategorií v~anglickém jazyce.

\begin{figure}[h]
\centering
\includegraphics[scale=0.6]{images/3v}
\caption{Schéma popisující 3V pomocí množinových diagramů \cite{3vimg}}
\label{fig:3v}

\end{figure}

\subsection[3v-volume]{Obsah (Volume)}
Data se dnes zdaleka nevyskytují jen v~textové podobě, můžeme je uchovávat formou hudby, obrázku či videa. Vzhledem k~tomuto faktu čelíme exponenciálnímu nárůstu množství uchovávaných dat a není výjimečné, aby enterprise systémy uchovávaly terabyty nebo petabyty dat. Data tedy tvoří množství informací, které je dost často vyhodnocováno z~různých úhlů a následně uloženo a znovu vyhodnocováno, a přestože původní data zůstala nezměněna, díky reevaluaci nám jejich množství roste závratným způsobem. Tímto způsobem může být na objem nahlíženo jako na jednu z~charakteristik BigData.

\subsection{Rychlost (Velocity)}
Na rychlost můžeme nahlížet hned ze dvou pohledů. Prvním pohled vyjadřuje rychlost, jakou nám data přibývají a jak aktuální pro nás jsou. Například historie vývoje měnového kurzu je informace, jejíž včerejší hodnota je naprosto nevypovídající a mění se s~každou minutou. Změnila se i rychlost, jakou noviny a televizní stanice získávají informace skrze sociální sítě. Objem dat tedy roste rychle a aktuálnost informací se rapidně zkrátila. Druhý pohled na tuto charakteristiku se zabývá rychlostí, jakou data potřebujeme zpracovávat. Jsou informace, které k~nám proudí velice často (například každou minutu), ale jejich vyhodnocení dává smysl ku příkladu jen jednou za 24 hodin. Ovšem jsou také data, které potřebujeme zpracovávat v~reálném čase, tak, jak k~nám proudí. Dobrým příkladem takových dat mohou být akutální informace z~meteorologické stanice. 
Rychlost, v~jaké jsou dnes data zpracovávána, se mnohonásobně zmenšila a tedy nejen objem dat, ale i rychlost zpracování reprezentuje samotná BigData.

\subsection{Různorodost (Variety)}
Jak je z~Obrázku ~\ref{fig:3v} patrné, různorodost dat znamená jejich strukturovanost/nestrukturovanost. Již bylo zmíněno, že data mohou mít mnoho podob. Ale i data ve stejné podobě (například textové), mohou být jinak strukturována a tomuto faktu je potřeba se přizpůsobit a uchovávat a zpracovávat data v~jiných formátech.
Různorodost a adaptabilita jsou tedy posledními charakteristikami BigData.

\section{BigData zjednodušeně}
Předcházející řádky by měly sloužit jako shrnutí a lehký úvod do problematiky BigData. Dalo by se vlastně také říci, že BigData nejsou jen o~velkém množství dat, ale je to celý koncept uchovávání a možností nových náhledů na stávající data, ale také návod, jak zachytit a zpracovat budoucí data. Další odstavce budou věnovány historii tohoto konceptu, ale také nejtypičtějších odvětví, kde se s~ním můžeme setkat. 

\section{Historie}

Již od počátků počítačové éry bylo potřeba data analyzovat. \cite{history} S~rychle se zvyšující dostupností moderních technologií a jejich obecnému přijetí ve společnosti, se posouvaly hranice této potřeby od vládních organizací až po současnost, kdy obrovské množství informací a jejích analýzu potřebují i malé podniky.

\subsection{30. a 40. léta}
V~této době se používaly první počítačové simulace. Prim hrálo především válečné odvětví, kde například vědci z~projektu Manhattan pomocí počítačových simulací simulovali dopad a následný zničující efekt jaderné bomby.

\subsection{50. a 60. léta}
V~tomto období se počítače (a s~nimi také zpracování a analýza dat) rozšířily do velkých korporací a výzkumných laboratoří. Počítač ENIAC například generoval první modely pro předpověď počasí. Analytici také vyřešili první problém nejkratší cesty a mnoho dalších, viz. také \cite{history}.

\subsection{70. až 90. léta}
V~této době se analytická činnost rozšířila o~středně velké podniky a technologické startupy. Objevují se také dnes již dobře známé případy užití. Dobrým příkladem takových případů je třeba první predikční model na pokles a růst akcií. Dále také stojí za zmínku první komerční nástroj pro modelově řízené rozhodování. Důležitým milníkem je rovněž vznik společností, jako je Ebay a Amazon. Bitva o~personalizaci online nákupů právě začala! Google implementuje první vyhledávací algoritmus, který zvyšuje relevanci výsledků.

\subsection{2000 až součanost}
V~tomto období se analytika rozšířila až na oblast malých podniku a analytických expertů (jednotlivců). Začíná mít obrovský dopad na život každého z~nás. Samozřejmostí začínají být dynamické změny cen zboží, doporučování produktů, hudby a filmů nebo řízení dopravy. Rozvíjejí se obory, jako je analýza a procesování přirozeného jazyka z~novin, e-mailů nebo sociálních sítí. Příchodu BigData, vzhledem k~levné dostupnosti výpočetního výkonu a rychlosti zpracování dat, již nic nestojí v~cestě.

\subsection{Budoucnost}
Předpokládá se, že v~budoucnu bude analytická činnost řídit každodenní rozhodování i na úrovni jednotlivců. V~běžném životě se přínos analýzy dat projeví například: Predikce v~policejní sféře a boji proti zločinu, výzkum ve zdravotnictví nebo kompletně personalizovaná zákaznická interakce i pro malé podniky a řetězce.

\section{Nástup sociálních sítí}
BigData zažila obrovský nástup také díky příchodu a masivnímu rozšíření sociálních sítí, a to hned ze dvou důvodů. Prvním důvodem je, že nástup sociálních sítí přilákal tisíce výzkumníků, kteří začali sbírat data z~Facebooku a Twitteru. Tito výzkumníci následně hledali různá spojení mezi zprávami a účty, z~kterých poté vyvozovali závěry ohledně těchto sociálních sítí. Další možností, k~čemu vytěžená data používali, bylo vytváření tzv. sociálních grafů. Historicky sbírali antropologové a sociologové data o~lidských vztazích skrze dotazníky, rozhovory, pozorování a experimenty. Dolováním dat ze sociálních sítí, kde lidé sdílejí mnoho detailů ze svých životů, se jim otevřel nový kanál, kde mají všechny tyto informace jednoduše k~dostání a stačí je pouze analyzovat.

V oblasti sociálních sítí existují 2 konkrétní druhy, které jsou z hlediska konceptu BigData zajímavé: \uv{Artikulované sítě} a \uv{Behaviorální sítě}. První kategorie znázorňuje sítě, kde uživatelé zadávají svá přátelství a konexe skrze technické mechanismy jako například: telefonní seznamy, emaily, seznamy přátel z~jiných sití atd. Druhou kategorií jsou sítě odvozené od komunikačních vzorců. Do této skupiny spadají uživatelé, kteří si píšou zprávy nebo jsou označení na společných fotkách. Obě tyto skupiny mají pro výzkumníky velký význam, přestože jím nepřikládají takovou váhu jako reálným osobním vztahům. \cite{social} 

Druhým důležitým aspektem, proč jsou sociální sítě pro BigData důležité, je fakt, že tyto sítě samy potřebují někde svá data uchovávat a zpracovávat. Technologické týmy stojící za těmito službami se tak ve značné míře podílejí jako kolaborátoři na BigData projektech či dokonce vytvářejí a následně uvolňují svoje technologie k~užití pro širokou veřejnost. Pro komunitu jsou důležité i přednášky a prezentované poznatky od těchto datových gigantů, kteří prozkoumávají a prolamují lidstvu dosud známe bariéry a umožňují tím využívání technologických pokroků i jiným subjektům. 

Sociální sítě samozřejmě nejsou jediným průkopníkem na poli BigData. Internetoví giganti jako Google, Amazon a Yahoo přispívají neméně výrazným dílem. Na sociálních sítích je však zajímavé to, že jejich data jdou do jisté míry jednoduše dolovat, díky čemuž vzniklo mnoho spolčeností, které se začaly jejich analýzou a sběrem zabývat a způsobily tím popularizaci spojení BigData se sociálními sítěmi. 


\section{Odvětví}

Jak již bylo zmíněno v~předchozí sekci, v~dnešní době můžeme na BigData narazit kdekoliv. Zde bych chtěl poukázat na široké spektrum využití napříč různými činnostmi, kterými se lidstvo zaobírá.\cite{sektory}

\subsection{Maloobchod}
Péče o~zákazníka a samozřejmě také zvýšení zisku jsou hlavními motivy pro zpracovávání a analýzu dat. Na základě chování uživatelů (aktivita na webu, zákaznická karta, anonymní zákazníci) můžeme předpovídat chování zákazníka v~každém stádiu nákupu. Toto chování lze navázat také na podniková data a hledat korelace pomocí Map Reduce mechanismů. Největším průkopníkem spojování BigData a maloobchodu je bezesporu řetězec Tesco se svou věrnostní kartou ClubCard. Na základě zákazníkovy nákupní historie sestavují žebříček produktů k~doporučení či dokonce dokáží odhadnout období těhotenství svých zákaznic.\cite{tesco}

\subsection{Věda a výzkum}

Není žádným překvapením, že ve vědě a výzkumu se využívají BigData na uchovávání výsledků z~měření či pro hledání korelací v~naměřených hodnotách. Držitel Nobelovy ceny Peter Higgs používal NoSQL databázový systém Cassandra na zpracování svých dat, díky nímž prokázal existenci tzv. Higgsova Bosonu \cite{higgs}.

\subsection{Meteorologie}
Díky sběru a vyhodnocování dat z~meteorologických stanic se podařilo vytvořit mnohem spolehlivější a přesnější modely pro předpovědi počasí, a to jak dlouhodobých, tak také krátkodobých. 

\subsection{Finance}
Ve finančním sektoru je způsobů využití hned několik. Například již výše zmíněné doporučování produktů dle historie transakcí a sběru  osobních dat. Banky a jiné finanční instituce podobným způsobem nabízejí zákazníkům vhodné finanční produkty, jako jsou například hypotéky. Mnohem zajímavějším případem využití BigData je detekce podvodů, kdy jsou banky na základě analýzy všech transakcí schopny hledat vzory podvodných chování a vyhodnotit určité transakce jako podezřelé a tím tak chránit své klienty nebo samy sebe.

\subsection{Webová optimalizace}
Na základě ukládání a následného zpracování veškerého chování uživatele na stránce mohou firmy optimalizovat webové stránky a jejich obsah či ho případně restrukturalizovat. Po vyhodnocení chování konkrétních uživatelů je možné jim obsah stránky automaticky personifikovat a podsouvat tak uživatelům pro ně zajímavé věci, aniž by se k~nim museli složitě proklikávat.

\subsection{BioInformatika}
V~bioinformatice se BigData využívají například k~mapování genomů nebo k~sekvenční analýze. Tyto informace pomáhají k~lepšímu pochopení DNA a také k~prevenci genetických poruch a vrozených nemocí či k~usnadnění jejich léčby. \cite{industries} 

\section{Dnešní možnosti}
V~dnešní době existují v~podstatě 3 možnosti jak začít s~BigData. 

\subsection{Specializované firmy a hotová řešení}
Na internetu nalezneme několik firem zabývajících se analýzou vašich dat, kde veškerá analýza a vizualizace probíhá v~softwaru třetí strany. Mezi nejznámější patří společnost Good Data \cite{gooddata}. Tyto firmy se však specializují na zpracování firemních dat a vizualizaci v~jejich vlastních BI nástrojích.

\subsection{Hotová enterprise řešení}
Další možností je vybrat nějaké komplexní řešení od firem zabývajících se platformou BigData, které vám dodají software pro ukládání, analýzu a vizualizaci vašich dat. Programování komponent či jejich konfigurace je v~režii zákazníka a tyto firmy poskytují licence, školení a technickou podporu. Tuto možnost poskytuje například IBM.

\subsection{Open Source a řešení z~něj vycházející} 
Poslední možností je použití Open Source nástrojů, které umožní ukládat, analyzovat a vizualizovat data. Toto je cesta, kterou jsem se vydal v~rámci této práce a budu se jí tak nadále věnovat. Drobnou nadstavbou těchto řešení mohou být firmy nabízející komerční balíčky těchto Open Source řešení. Jedná se o~velice populární přístup v~případech, kdy je potřeba zkombinovat několik nástrojů dohromady. Jejich konfigurace bývá velice obtížná, a proto tyto komerční balíčky nabízí již nakonfigurovaná hotová řešení většinou i s~drobnou nadstavbou, která umožňuje provádět některé nadstandardní procesy a činnosti. 



%mozna jen podsekce bigdata%
\chapter{BigData techniky}


Abychom mohli hovořit o konkrétních technologiích, musíme nejdříve definovat základní technologické stavební kameny, o které se BigData opírají. Jedná se o technologie a paradigmata vyvinutá technologickými giganty, kteří jako první naráželi na technologické hranice a posouvali je dál. Pořadí těchto technologií a paradigmat se odvíjí od logické posloupnosti tak, jak spolu souvisí a navazují na sebe.    

\section{Distribuované systémy}
Distribuované systémy jsou tématem, které zasahuje samozřejmě mnohem dále než jen do oblasti BigData. Pro zjednodušení následujících řádků zavedeme následující rozdělení: Distribuované systémy rozdělíme na systémy s distribuovaným výpočetním výkonem, systémy s distribuovaným úložištěm anebo kombinaci obojího. 

Distribuovaný výpočetní výkon je takový systém, kde se výpočet jedné úlohy rozloží na více počítačů. Paralelní výpočty jsou známé z oblasti informačních technologií již mnoho let a používají se nejčastěji k vědeckým výpočtům, paralelní kompilaci zdrojových kódů anebo k jiným operacím, které by na jednom počítači trvaly příliš dlouho. 

Distribuované úložistě je jednoduše představitelné a důvodů mít data na více místech je hned několik.
\begin{itemize}
\item Záloha - i ze sféry osobních počítačů známe trend, kdy máme data zálohovaná na více fyzických zařízeních, abychom předešli jejich ztrátě v případě technické poruchy na daném zařízení. 

\item Nedostatek kapacity - Ze sféry osobních počítačů známe, že uživatelé méně potřebné soubory ukládají na externí periferie, protože kapacita disků v osobních počítačích je zpravidla v řádu stovek GB.

\item Dostupnost - Pokud chce uživatel mít přístup k jednomu souboru z domácího i pracovního počítače, musí se tento soubor fyzicky nacházet na obou počítačích nebo musí využít nějaký software na sdílení souborů. 

\end{itemize}

Kombinovaný přístup je zřejmý. Používá se jak výpočetní výkon jednotlivých počítačů v systému, tak jejich úložný prostor. 

BigData využívá všechny tyto přístupy. Je zřejmé, že obrovské množství dat, o kterém jsme mluvili v úvodu, se lépe zpracovává na více počítačích. Stejně tak je logické, že velké množství dat uložíme na více počítačů a to platí i v případě, že chceme data zálohovat. 

Ne vždy však využíváme kombinovaný přístup. Je totiž velmi časté, že si firmy staví obrovské počítačové farmy, které slouží pouze jako datové sklady a datová centra. Vždy záleží na konkrétní situaci a způsobu využití. 


\section{CAP Theorem}
 Při nástupu distribuovaných systémů v roce 2000 vydal vědec Eric Brewer článek popisující tzv. CAP Theorem \cite{cap}, který se distribuovaných systémů přímo týká. Tento teorém říká, že distribuované systémy mají tyto 3 hlavní vlastnosti: 

\begin{itemize}
\item Konzistence (Consistency) - Vlastnost, která určuje, zda pro každý požadavek vrátí server správný výsledek. To znamená, že odpověď je adekvátní vůči specifikaci požadované služby. Přesný význam konzistence se odvíjí od typu služby. V případě dat ji definujeme tak, že každý server má aktuální a stejná data.

\item Dostupnost (Availability) - Vlastnost, která říká, že na každý požadavek dostaneme odpověď. Rychlejší odpověď je preferovanější oproti pomalejší, ale v kontextu teorému je důležité, že odpověď vůbec dorazí. Z praxe však víme, že velice opožděná odpověď je stejně špatná jako žádná odpověď. Můžeme tedy tuto vlastnost zjednodušit a říci, že systém musí být vždy dostupný.

\item Tolerance výpadku (Partiotion tolerance) - Tato vlastnost (jako jediná) určuje chování podpůrného systému, na kterém služba běží, namísto popisu chování služby samotné. Tato vlastnost říká, jestli během výpadku nějaké části systému je tento systém schopný pokračovat a dále fungovat.

\end{itemize}


Brewer popisuje, že každý distribuovaný systém může splňovat nejvýše 2 z těchto 3 vlastností.

\subsection{CAP Theorem v roce 2012}
V roce 2012 napsal Brewer další článek \cite{cap2}, ve kterém popisuje stav jeho teorému po 12 letech. Vysvětluje, že již od začátku bylo označení \uv{pouze 2 ze 3} zavádějící a vágní, protože spoustu věcí příliš zjednodušovalo. Například u systému s velkou granularitou se mezi volbou C a A rozhoduje na několika úrovních a všechny vlastnosti mají spíše hodnoty v čase a ne hodnoty binární a dále také, že záleží na stavbě systému a jeho drobných nuancích. Zároveň však píše, že ve svém důsledku teorém splnil svůj účel a otevřel systémovým návrhářům oči při navrhování distribuovaných systémů a donutil je se zamyslet nad výhodami a nevýhodami jednotlivých vlastností systému. 

V témže roce vyšel další zajímavý článek popisující aktuální stav CAP teorému \cite{cap3}. Popisuje převážně vztah
systémů k volbám CAP vlastností. 

\subsubsection{Nejlepší možná dostupnost}
Nejčastějším výběrem je garantovaná konzistence s maximální možnou dostupností. Pro většinu systémů je to přirozená volba. Server tedy  za každou cenu vrací správnou odpověď a snaží se poté optimalizovat co nejvyšší dostupnost a nejkratší možnou dobu odpovědi vzhledem k síťovým podmínkám. Tento přístup dává největší smysl, pokud jsou počítače ve stejném datacentru a běží na nich stejná služba. Typickým zástupcem je \uv{lock service} a služba spravující metadata pro nějaký distribuovaný systém s nízkou granularitou.

\subsubsection{Nejlepší možná konzistence} 
Druhou nejčastější skupinou jsou systémy, pro které je ztráta dostupnosti nemyslitelná a tudíž ji garantují a snaží se o co nejvyšší úroveň konzistence. Tento postup nejlépe vyhovuje v situacích, kdy máme počítače distribuované napříč několika datacentry. V tomto případě může totiž dostupnost rapidně klesat s jakoukoliv chybou, a proto je potřeba ji garantovat. V těchto případech tedy designéři obětují konzistenci, aby mohli garantovat dostatečně rychlou odpověď, přestože ta nemusí být vždy zcela správná. Ideálním příkladem jsou webové cache a obrázkové servery.

\subsubsection{Segmentovaná konzistence a dostupnost}
Toto je nejzajímavější možnost a pro tuto práci je stěžejní. Existují systémy, které nemají jednotné požadavky pro všechny aspekty služby. Některé vyžadují silnou konzistenci a některé vysokou dostupnost. Pro dodržení CAP teorému se jako nejpřirozenější možnost jeví rozdělit systém na několik jednotlivých komponent, které budou specificky nastaveny. Tím pádem celý systém nezaručuje ani konzistenci, ani dostupnost, ale každá část systému poskytuje vlastnosti, které potřebuje. Segmentace může probíhat na několika možných úrovních: 

\begin{itemize}
\item \textbf{Rozdělení podle dat} - Jiná data mohou vyžadovat jinou úroveň dostupnosti a konzistence.
\item \textbf{Rozdělení podle operací} - Operace pro zápis mohou mít jiné požadavky na konzistenci a dostupnost než operace pro čtení.
\item \textbf{Rozdělení podle funkcí} - Některé služby mohou být rozděleny na podslužby a pro každou takovouto službu můžeme mít vlastní úroveň konzistence a dostupnosti. 
\item \textbf{Rozdělení podle uživatelů} - Jedná se o rozdělení závislé převážně na geografické poloze uživatele. Služba může pro uživatele, který se nachází blízko, zaručit vysokou dostupnost a zároveň v rámci jemu blízkého datacentra udržovat i konzistenci. 
\item \textbf{Rozdělení podle hierarchie} - Jedná se o systém, ve kterém se na určitých úrovních kombinují výše popsaná rozdělení.
 
\end{itemize}
\section{Distribuovaný file systém}

V předchozích sekcích jsme rozebrali distribuované systémy a jejich omezení. Zmínili jsme také nutnost sdílet velké množství dat napříč několika počítači, které mohou být umístěny napříč různými datacentry. Potřebu mít distribuovaný filesystém měl i jeden z největších technologických gigantů – firma Google. V roce 2004 se Google rozhodl o jejich řešení podělit a vydal detailní článek \cite{gfs} popisující kompletní funkčnost a detaily celé infrastruktury. Vysvětlení funkčnosti a architektury je nad rámec této práce a navíc je vše dobře popsané v článku samotném, důležité je však zmínit, že tento systém se stal inspirací a nastolil trend v tom, jak podobné systémy dnes vypadají a jaké mají vlastnosti. Na základě tohoto článku vznikl například opensource klon MooseFS.

\begin{figure}[h]
\centering
\includegraphics[scale=0.5]{images/gfs}
\caption{Architektura Google File System \cite{gfs}}
\label{fig:3v}
\end{figure}

\section{BigTable}
Jedná se o systém na uchovávání dat společnosti Google, který je postavený nad Google File Systémem a jinými technologiemi společnosti Google. Jedná se o proprietární software, který není k dispozici mimo firmu Google vyjma možnosti využívat tento software jako část služby Google App Engine. V roce 2006 opět Google zveřejnil článek o BigTable \cite{bigtable}, avšak nyní s mnohem méně detaily nežli u svého článku o GFS \cite{gfs}, zřejmě z obav o přesné zkopírování jako u zmíněného systému. V době vydání článku BigTable obsluhoval více než 60 služeb firmy Google a škáloval několik petabytů dat na několika tisících počítačích. BigTable umožňuje využívání MapReduce frameworku. Jedná se o první implementaci mnohasloupcové, distribuované, multidimenzionální perzistentní seřazené mapy. Google opět udal trend, jakým se podobné systémy začaly ubírat. Datový model BigTable v budoucnu inspiroval i tvůrce Apache Cassandra, která je hlavním tématem této práce a proto detaily ohledně datového modelu prozatím přeskočíme.

\section{MapReduce}
Je programovací model a framework prvně představený společností Google \cite{mapreduce} na zpracování velkého datasetu. Uživatel specifikuje mapovací funkci, která zpracuje páry (klíč, hodnota) a vygeneruje přechodné páry (klíč, hodnota), které jsou pak předané redukční funkci, která sloučí všechny mezi-hodnoty se stejným mezi-klíčem. Mnoho příkladů z reálného světa lze převést do tohoto paradigmatu. Výhodou těchto funkcionálně napsaných programů je, že jsou automaticky dobře paralelizovatelné. Systém se stará o detaily distribuce dat a rozhození jednotlivých úloh na jednotlivé počítače a obsluhuje chyby a neočekávané stavy. Díky tomu umožňuje i programátorům s velmi malou znalostí paralelního programování napsat vysoce paralelní a efektivní programy. 

Toto paradigma se stalo nejdůležitějším stavebním kamenem pro zpracovávání BigData. Díky tomuto mechanismu jsme schopni zpracovávat terabyty dat rychle a efektivně a vypočítaný výsledek znovu uložit do databáze. Veškeré postupy zpracování BigData jsou přímo či nepřímo založené na MapReduce. 

\begin{figure}[!h]
\includegraphics[scale=0.6]{images/mapreduce}
\caption{MapReduce schéma \cite{mapreduce}}
\label{fig:mapreduce}
\end{figure}

\newpage

\section{NoSQL}
Se zkratkou SQL se můžeme setkat u relačních databázových systémů. Zkratka NoSQL neznamená pravý opak, písmena \uv{No} znamenají \uv{Not Only}, tedy ne pouze. Jedná se o databázový koncept, který se vyskytuje u nerelačních databází. V tomto konceptu datové úložiště i zpracování dat používají jiné prostředky, než je běžné tabulkové schéma relační databáze. Výhodami tohoto konceptu jsou jednoduchý design a horizontální i vertikální škálovatelnost. NoSQL databáze často podporují také podmnožinu jazyka SQL. Většinou se jedná o jednoduché funkce, jako je vkládání a velice jednoduché výběry. Některé NoSQL databáze mají i velice odlišný ukládací model (například stromový či grafový), tím pádem je složitost pro různé operace odlišná. Nejčastější podobou NoSQL databáze jsou dvojice klíč-hodnota, čili mapa. Podle dosavadních informací lze říci, že Google BigTable je NoSQL databáze. Mezi charakteristiky NoSQL databází můžeme zahrnout

\begin{itemize}
\item \textbf{Datový a dotazovací model} - Jak již bylo řečeno, NoSQL databáze se liší způsobem udržování dat a dotazováním nad nimi.
\item \textbf{Perzistence} - Ne všechny NoSQL databáze ukládají svá data na disk, některé databáze běží pouze v operační paměti.
\item \textbf{Rozhraní} - Některé databáze komunikují skrze REST rozhraní a některé pomocí binárních protokolů. 
\item \textbf{BASE} - Tak jako relační databáze využívají vlastnosti ACID (Atomic Consistent Isolation Durability), tak v NoSQL je ekvivalentem BASE (Basically Available, Soft state, Eventual consistency), kde každá NoSQL databáze garantuje jednotlivé vlastnosti různými mechanismy a nastaveními nebo je databáze již od základu navržena s danými vlastnostmi.   
\end{itemize}


NoSQL databáze jsou dalším základním stavebním kamenem BigData. NoSQL databáze jsou často kompatibilní s MapReduce konceptem, čímž tvoří ideální dvojici pro uchovávání a zpracování velkého množství dat. 



\chapter{BigData Platforma Apache}

V předchozí kapitole jsme se seznámili s technickými principy a postupy, které se uplatňujív BigData konceptu. V tét kapitole se budu zabývat jedním z hlavních témat práce a tou je open source řešení pro platformu BigData od organizace Apache Foundations, také známý jako Apache Big Data stack. Tento stack se skládá z několika aplikací, které až na jednu výjimku na sobě nejsou nikterak závislé. V záběru této kapitoly se budu důkladně věnovat pro nás nejdůležitější části tohoto stacku a tím je NoSQL databázový systém Cassandra.

\section{Hadoop}




Hadoop je softwarová knihovna napsaná v programovacím jazyce Java a umožňuje distribuované zpracování velkého množství dat napříč clusterem pomocí jednoduchých programovacíh modelů- Je navržený aby dobře škáloval cluster tvořící jeden až několik tisíc počítačů, kde každý nabízí lokální výpočetní výkon a úložiště dat. Hadoop řeší problémy s hardwarem na aplikační vrstvě a tudíž je možné navrhovat vysoce dostupné služby na clusterů počítačů, aniž bychom se museli strachovat výpadků. Jeho 4 komponenty tvoří nejpodstatnější část pro celou analytickou práci s  daty. A všechny ostatní aplikace přímo, čí nepřímo některé části Hadoopu používají nebo jsou na nich dokonce založeny. Jaké tedy jsou tyto 4 komponenty hadoopu ?


\begin{figure}[h]
\centering
\includegraphics[scale=0.15]{images/hadoop}
\caption{Logo Apache hadoop}
\label{fig:yarn}

\end{figure}


\subsection{Hadoop Commons}
Jedná se pouze o základní sadu nástrojů podporujícíc a propojující ostatní moduly Hadoopu.


\subsection{Hadoop File System (HDFS)}
Jedná se o open source distribuovaný filesystém, jehož některé prvky vychází z dříve zmíněného GFS. HDFS umožňuje uložit velké množství dat mezi jednotlivé uzly.  HDFS umožňuje velice dobré škálování s rostoucím objemem dat. Ostatní technologie z Apache Big Data Stacku filesystém využívají ke sběru a ukládání výsledků jejich analytických procesů. HDFS cluster se skládá primárně z NameNode, který řídí filesystemová metadata a z DataNodů, které data uchovávají. 

\subsection{Hadoop MapReduce}
MapReduce jsme zmínili výše a jen zopakuji, že se jedná o programovací paradigma a framework na paralelní zpracování dat. Nabízí API, díky kterému můžeme jednoduše programovat naše vlastní MapReduce operace. Tento framework nabízí základní kostru, kterou programátor doplní a o vše ostatní se stará samotná knihovna. Přesto všakveškerá logika programu zavisí na programátorovi. 

\subsection{Hadoop YARN}
Tento modul se stará o plánování jednotlivých MapReduce programů a o správu dostupných zdrojů v celém clusteru a rozhoduje jaká data se kam budou posílat a počítat. Základní Architektura YARNu má za myšlenku mít jeden globální uzel, který se nazývá \uv{Resource Manager} a Pro každý běh aplikace mít tzv. \uv{Application Master}, který má na starosti komunikaci s \uv{Node Managery} a dohlížením nad spouštěním jednotlivých Tasků. Resource Manager se skládá ze 2 hlavních komponent Plánovače a Aplikačního manažera. Plánovač je zodpovědný za alokování zdrojů pro různé běžící aplikace. Aplikační manažer je zodpovědný za příjem nových MapReduce programů a jejich správné zařazení. NodeManager je agent běžící na každém stroji, která je zodpovědný za aplikační kontejnery a monitoruje stav dostupných prostředků stroje a ohlašuje se Resource Manageru, který tak má potřebné informace k přerozdělování nových úkolů.

\begin{figure}[h]
\centering
\includegraphics[scale=0.7]{images/yarn_architecture}
\caption{Architektura Hadoop YARN}
\label{fig:yarn}

\end{figure}



\newpage 

\section{HBase}
Je NoSQL databázový systém, který je založený na principu Google BigTable. Jedná se o druhou nejpoužívanšjí NoSQL databázi v Apache Big Data Stacku. Implementačně je zavíslá na HDFS, který používá k ukládání dat. Výhodou je velice jednoduhá a rychlá integrace s Hadoop MapReduce. HBase sdědil po Hadoopu i jeho Master-Slave architekturu.


\section{Hive}


Apache Hive je data warehouse infrastruktura postavená na vrchu Hadoopu na poskytování analýzy a dotazování se nad daty. Původně byl tento software vyvinut ve společnosti Facebook. A nyní se jeho starají společnost jako je NetFlix a Amazon. Hive podporuje anaýzu velkých datasetů uložených v HDFS nebo HDFS-kompatibilních systémech jako například (CassandraFS nebo Amazon S3). Syntaxe jeho dotazovacího jazyka HiveQL je velice podobná SQL a proto ho mohou inženýři ovládající tento jazyk velice brzy ovládnout.HiveQL nabízí programátorům prostředky, které nejsou běžně v NoSQL databázích dostupné jako například funkce JOIN. 

\begin{wrapfigure}{r}{0.5\textwidth}
  \centering
    \includegraphics[scale=0.5]{images/hive_logo}
\caption{Logo Apache Hive}

\end{wrapfigure}

Hive funguje na principu, že si dotaz přeloží a převede do kodu kompatibilního s Hadoop Mapreduce a výsledný program následně spustí na Hadoop clusteru. Obrovskou výhodou je, že programátor dostává v podstatě s nulovým snažením hotový MapReduce program, který by jinak musel zdlouhavě psát, což umožnuje být vysoce efektivní. Hive si standardně ukládá metadata do embedované databáze Derby. Podporuje pokročilejší funkce jako například Indexy, kompresi nebo uživatelem definované funkce (UDF) samozřejmostí jsou standardní matematické a řetězcové funkce aplikovatelné v HiveQL dotazech. 



\section{Pig}


Pig funguje na stejném principu jako Hive. Tedy , že převede vámi napsaný kód do hotového MapReduce řešení a tento kód spustí nad Hadoop clusterem. Tato knihovna byla napsána ve firme Yahoo a později opensourcována pod Apache foundation. Pig používá svůj vlastní jazyk pojmenovaný Pig Latin a oproti HiveQL nemá s SQL moc společných rysů. Latin se spíše podobá funkcionálním jazykům. Pig a Hive dělají tedy stejné věci, tak proč používat obojí? Je to spíše volba osobních preferencí, či preferencí vašeho vývojářského týmu. S oběma technologiemi lze dosáhnout stejných výsledků. Přesto má Pig lepší uplatnění v případě tvorby komplikověnšjích data flow a HiveQL se více hodí pokud jsou potřeba Ad-Hoc dotazy. Přestože lze dosáhnout stejných výsledků s oběmi platformami, určitě doporučuji umět obě a použít Hive nebo Pig v konkrétních siuacích, kde mají lepší užití. 



\section{SolR}

Solr [Solar] je vyhledávací platformou odvozenou od Apache Lucene. Mezi její hlavní výhody patří:

\begin{itemize}
\item Pokročilý full-text vyhledávací engine
\item Optimalizace pro vysokoobjemový tok dat
\item Otevřené rozhraní pomocí XML,JSON nebo HTTP
\item Lineární škálovatelnost, automatická replikace indexu, auto failover a samoobnovení
\item Indexace v témeř reálném čase
\item Možnost doprogramování vlastních pluginů
\end{itemize}

Solr využívá Lucene index, což znamená, že je formát je striktně definovaný. Změna ve formátu indexu znamená reindexaci celého dokumentu. Díky lehkému nastavení a možnostem výstupu a pokročilými funkcemi jako je například GeoSpatial vyhledávání nebo facetové full text hledání, je SolR v komerční a open source sféře vyhledávacím enginem číslo 1.


\section{Squoop}

Tento nástroj je velice užitečný, protože ne vždy chceme mít všechna data pouze v NoSQL databázích a pokud to je možné, využití výhod relačních databází je logickou volbou. Apache Squoop je nástroj pro hromadnou migraci z HDFS (či kompatibilního) file systému do strukturovaných úložišt jako například: relačních databází. Přesun dat je navržený obousměrně stejně tak můžeme využít Squoop k migraci dat z  relačních databází do HDFS (a kompatibilních) systémů. Výhodou je vysoká efektivita a nízká časová investice oproti psaní vlastních migračních scriptů.

\section{Mahout}

Mahout je slovo, které pochazí z Hindi a jeho překladem do češtiný získáme spojení jezdec na slonovi. Slonem se v BigData komunitě myslí projekt Apache Hadoop, který má slona ve svém logu a je s tímto zvířetem neodmyslitelně spojený. Projekt Mahout se snaží svým metaforickým názvem naznačit využívání hadoopu k vytvoření knihovny pro podporu škálovatelného \uv{Machine learningu}. Mahout je sada funkcí a algoritmů využívaných v odvětví machine learningu, naprogramovaných pomocí Hadoop MapReduce paradigmatu. Primární zaměření Mahoutu je na odvětví kolaborativního filterování, clusterování a klasifikace dat. Mahout také obsahuje vysoké množství matematických funkcí a algoritmů z oblasti lineární  algebry a statistiky. V současné době lze využít Mahout na tyto základní způsoby užití. Doporučovací funkce na základě uživatelského chování, rozřazování dokumentů do clusterů na základě shody v obsahu a Klasifikace dokumentů na základě již uložených kategorizovaných dokumentů. 

\section {ZooKeeper}

Autoři projektů v rámci Apache foundations a především ti zaměření na projekty BigData stacku mají pro metafory a různá symbolická spojení velkou slabost. Jak názvy a loga projektů napovídají, velká část projektů má v názvu nebo logu nějaké zvíře, či referenci na něj. Zároveň uhlídat a spravovat byť malý ale distribuovaný systém, může být velký chaos a proto je potřeba ten správný hlídač. ZooKeeper v překladu tedy hlídač zoo je nástroj pro správu a konfiguraci distribuovaných systému z platformy Apache BigData. 



\chapter{Cassandra}	


V předchozí kapitole jsme rozebrali software z apache Big Data Stacku kromě jednoho. Cassandra je NoSQL WideColumn databáze a jedná se o nejdůležitější software z Apache BigData stacku z pohledu této práce, která se má na Cassandru zaměřit. Z tohoto důvodu tuto databázi do detailu rozebereme. 

\section{Proč Cassandra}
Přestože NoSQL databází vhodných k uchovávání a zpracovávání BigData existuje celá řada, Cassandru jsem si vybral, protože si myslím, že je nejzajímavější z hlediska architektury, dosahuje skvělých výsledků v porovnávacích testech\cite{benchmark}, používají ji v produkci a velikých clusterech známé společnosti, má největší komunitu (a na vývoji se aktivně podílejí i velké firmy), a je nejpoužívanější databází e své kategorii\cite{dbengines} a v neposlední řadě také proto, protože jsem se s touto databází setkal již dříve a chtěl jsem zjistit kam se databáze za tu dobu dostala.

Díky široké nabídce funkcí a výše popsaným důvodům si myslím, že je tato databáze vhodná i pro edukační účely pro její jednoduchost a zároveň komplexnost. Vzhledem k jednoduchému začlenění s ostatním softwarem z Apache BigData stacku se stává z Cassandry ultimátní BigData nástroj. Vhod může absolventům přijít i fakt, že je Cassandra průmyslově nejpoužívanější databází svého druhu.


\section{Vznik}

Cassandra vznikla v roce 2008 ve společnosti Facebook kde sloužila jako úložiště zpráv mezi uživateli tété obří sociální sítě. Facebook později téhož roku uvolnil zdrojové kódy projektu. Projekt se v roce 2009 dostal do inkubační fáze projektu pod záštitou Apache Foundations. O rok později v roce 2010 se projekt dostal mezi hlavni plnohodnotné projekty Apache Foundations. Jak bylo již zmíněno velké rozdíly mezi verzemi jednotlivých projektů z Apache Big Data stacku způsobují velké problémy s kompatibilitou celého systému, Cassandra není výjimkou i její rozdílné verze nesou rizika a rozdílné funkce a samozřejmě nekompatibilitu. V současné době je nejaktuálnější verzí Cassandra 2.04 a pokud nebude uvedeno jinak, uvažujeme v textu tuto verzi. 

\section{Předchůdci}

Když Facebook vydal článek popisující Cassandru \cite{facebookcassandra} popsal ji jako \uv{To nejlepší z Google BigTable a Amazony Dynamo} což není náhodou jelikož prvotním šéfem vývoje Cassandry ve Facebooku byl právě inženýr, který ve společnosti Amazon navrhl a vytvořil databázi Dynamo. Podoba BigTable také není náhodou, Facebook pro své potřeby potřeboval wide-column databázi a proto inspirace BigTable byla přirozenou volbou. 

\subsection{Co zdědila od Dynama}
\begin{itemize}
\item \textbf{Symetrie} Každý uzel má stejnou zodpovědnost a roli v systému jako jakýkoliv jiný uzel.
\item \textbf{Decentralizace} Rozšíření symetrie. Každý uzel může plně komunikovat s jakymkoliv jiným uzlem a získávat informace a ovládat systém skrze jakýkoliv uzel přispívá k výborným škálovacím vlastnostem. 
\item \textbf{Heterogenita} Každý uzel může mít jiné vlastnosti např. kapacita disků a přidání nového uzlu s jinými vlastnostmi, by nemělo systém nikterak ohrozit 
\end{itemize} 

\subsection{Co zdědila od BigTable}
\begin{itemize}
\item \textbf{Datový model} Stejně jako BigTable má Cassandra Wide-Column schéma reprezentované klíčem a hodnotou.
\item \textbf{Distribuovaný log} K zápisu používá databáze interně distribuovaný \uv{append log}. 
\end{itemize} 

\section{Kdo Cassandru používá v produkci}

Cassandru v produkčním provozu můžeme nalézt u mnoho obrovských společností, které ji používají v Clusterech od pár uzlů až do největšího známého Cassandra clusteru o velikosti 400 uzlů a 300TB dat. Mezi nejznámější společnosti patří

\begin{itemize}
\item Twitter
\item Ebay
\item Spotify
\item NetFlix
\item Cisco
\item Urban Airship
\item Reddit
\item Accenture
\item a mnoho dalších...
\end{itemize} 

\begin{figure}[h]
\centering
\includegraphics[scale=0.45]{images/casa-production}
\caption{Ukázka ostatních firem využívajících Cassandru v produkčním prostředí}
\label{fig:yarn}
\end{figure}


\section{Společnosti vyvíjející Cassandru}
Cassandra je open source projekt, který ale vyvíjí dvě velké společnosti Acunu a Datastax. Především druhá jmenovaná společnost se o vývoj a budoucnost Cassandry stará nejvíce. Datastax pořádá školení a certifikuje vývojáře a administratátory, pořádá setkání a především  Cassandru přidává do svého komerčního BigData balíčku která je nadstavbou nad Apache BigData stack přidávající vlastní functionalitu a spojující všechny komponenty v jeden funkční celek. Pro nekomerční užití v testovacích prostředích je tento balíček k dispozici zdarma.
\section{Hlavní výhody}

Cassandra nabízí pro vývojáře mnoho lákadel pro její využítí zde vypíšu ty nejzajímavější, které postupně rozepíši. 

\begin{itemize}
\item \textbf{Replikace} Systém sám replikuje data podle kriterií, které mu zadáme.
\item \textbf{Transparentní škálování} Programátor může napsat kód proti lokální instanci a aplikace bude fungovat v jakémkoliv clusteru.
\item \textbf{Nastavitelná konzistence} Můžeme měnit nastavení konzistence za běhu aplikace
\item \textbf{Nastavitelná síťová strategie} Správce může definovat strategii podle které systém pozná, do jakého datacentra případně racku má uzel zařadit.
\item \textbf{Cassandra Query Language (CQL)} Dotazovací jazyk velice podobný SQL
\item \textbf{MapReduce} Podpora Hadoop MapReduce
\item \textbf{Databáze nemá \uv{Single Point of Failure}} Díky plně P2P architektuře není databáze  nachylná na výpadek jakéhokoliv uzlu.
\item \textbf{Rychlost} Čtení i zápis dosahuje skvělých výsledků \cite{benchmark} a to i při lineárním škálování. 
\end{itemize} 

\begin{figure}[h]
\centering
\includegraphics[scale=0.5]{images/netflix}
\caption{Škálování Cassandry a vliv na I/O operace}
\label{fig:scaleup}
\end{figure}

\section{Architektura}
Architektura Cassandry se skládá z několika klíčových bodů z nichž každý plní svou roli. V následujících podsekcích je stručně vysvětlím.

Cassandra je plně symetrický systém, kde neexistuje jediný bod selhání. Její architektura byla navržena s premisou, že podpůrný hardware a systém mohou a také padají a proto přináší symetrickou arhitekturu, kde jsou si všechny uzly rovny a veškerá data jsou uložena napříč všemi uzly v clusteru. Commit logy na každém uzlu zachycují veškeré zápisy a data jsou také uložena do mezipaměti, která když se zaplní, tak se data zapíší na disk a automaticky zreplikují a rozdistribuují po celém clusteru. 

Klientské požadavky na zápis nebo čtení mohou přijít na kterýkoliv uzel z clusteru a ten se pak pro daný požadavek stává koordinátorem, který se stává jakousi proxy mezi klientem a uzlem, který data vlastní. Tato architektura nám zajištuje lineární škálování viz \ref{scaleup}

\subsection{P2P protokol \uv{Gossip}}
Do češtiny bychom mohli název tohoto protokolu přeložit jako štěbetání nebo klábosení a tento název je trefný, protože to je v podstatě to, co servery opravdu dělají. plně Peer to peer architektura systému s protokolem kde si jednotlivé uzly o sobě sdělují všechny informace nám dává k dispozici systém, kde každý uzel ví, kde má hledat která data, který uzel je nedostupný a kterému uzlu by měl nějaká data poslat. Když do systému připojíme nový uzel ostatní uzly ho ihned začnou zahrnovat informacemi o všech ostatních uzlech v clusteru. To samé platí i pro uzly, které se do clusteru vrátí po nějakém výpadku. Každý uzel o sobě ostatním uzlům posílá informace každou vteřinu každý uzel pošle informaci o sobě a ostatních až 3 uzlům najednou a informace sebou nese i časové razítko, aby se předešle informace mohli přepsat. Tímto způsobem se velice rychle všechny clustery dozví vše o všech ostatních uzlech v clusteru. 

Pokud je uzel označený jako nefunkční Cassandra na něj neposílá žádně požadavky, stejně tak může neposílat požadavky na funkční, ale příliš vytížený uzel pomocí dynamického prohledávání. O zapisování dat o která uzel přišel po dobu svého výpadku rozebereme v jedné z dalších sekcí. 

\subsection{Rozdělovač}
Rozdělovače, anglicky (Partitioner) určují, jakým způsobem budou data a jejích repliky rozloženy po clusteru. Jednoduše řečeno jedná se o hashovací funkci, která vezme klíč, který chceme ukládat a vrátí nám token, podle kterého určíme, na které uzly se má tato hodnota zapsat. Na výběr máme ze 3 možností

\begin{itemize}
\item \textbf{Murmur3Partitioner} - jedná se o počáteční hodnotu, která rozmisťuje data rovnoměrně po clusteru na základě MurMur3 hashovací funkci.
\item \textbf{RandomPartitioner} - rovnoměrně umisťuje data po clusteru na základě MD5 hashovací funkce (je pomalejší než MurMur3)
\item \textbf{ByteOrderedPartitioner} -  Ukládá data po clusteru na základě lexikálního pořadí bytů. Nedoporučuje se, kvůli složitému vyvažování a nerovnoměrné distribuci dat. Jedinou výhodou je sekvenční hledání podobných klíčů.
\end{itemize}

\subsection{Distribuce a replikace dat}
Když si zvolíme vhodný rozdělovač na základě, kterého určujeme kam, příslušný klíč patří, musíme serverům přiradit určitou výseč klíčů. 

\subsubsection{Kruh}
V Cassandře uzly tvoří takzvaný kruh. Podle počtu uzlů v kruhu se vypočte výseč tokenů, které danému uzlu patrí. 

\subsubsection{Virtuální uzly}
Cassandry od verze 1.2 přináší závratné vylepšení v podobě virtuálních uzlů, kde se každý uzel rozdělí na několik virtuálních uzlů (každý dostane svou náhodnout výseč) a tím se rapidně snižuje počet tokenů, které uzlu patří. Virtuální uzly zlehčují a vylepšují v Cassandře hned několik úkonů.

\begin{itemize}
\item Tokeny se přiřazují samy a náhodně pro nové uzly již není nutné tokeny počítat a přiřazovat manuálně.
\item Oprava mrtvého uzlu je mnohem rychlejší, jelikož se naní podílí každý ostatní uzel v clusteru a změny probíhají inkrementálně.
\item Po přidání nového uzlu nemusíme dělat složité prevažování clusteru, virtuální uzly se vytvoří rovnoměrně mezi ostatními uzly a vezmou si tak od každého uzlu mnohem menší objem dat.
\item Zlepšuje využití heterogeních strojů v clusteru, můžeme strojům s rozdílnou kapacitou nastavit rozdílný počet virtuálních uzlů.
\end{itemize}

\begin{figure}[h]
\centering
\includegraphics[scale=0.5]{images/vnodes_compare}
\caption{Porovnání Kruhu s a bez virtuálních uzlů}
\label{fig:vnodes}
\end{figure}

Když má každý (virtuální) uzel přiřazenu svou množinu tokenů a máme určeno jak budeme tokeny získávat. Je potřeba vyřešit replikování dat. Počet replik je daný replikačním faktorem, který se nastavuje pro každou databázi. Replikační faktor 1 znamená, že každá řádka bude existovat pouze 1. Replikační faktor 2 znamená, že každá řádka bude uložena dvakrát a pokaždé na jiném stroji.  V Cassandře jsou si všechny uzly rovny a stejně tak každá replika dat, tedy neexistuje žádná hlavní replika, všechny jsou stejně důležité. Replikační faktor by neměl překročit maximální množství uzlů v clusteru, data budou redundantní a budeme ztrácet výkon. Na výběr máme ze dvou replikačních strategií.


\begin{itemize}
\item \textbf{Jednoduchá strategie} - Využívá se pouze pro clustery uložené v jednom datovém centru. Tato strategie uloží první repliku na uzel, určený rozdělovačem a ostatní repliky jsou uloženy na následujících uzlech v kruhu po směru hodinových ručiček bez ohledu na topologii (pozice v racku či datacentru)
\item \textbf{Síťová strategie} - Replikační faktor se zde změní na replikační faktor pro každé datacentrum. Každé datacentrum má vlastní kruh. Rozmisťování kopií tedy funguje následovně. První kopie se uloží na uzel vybraný rozdělovačem a další kopie ve stejném datacentru se uloží na nejbližší uzel po směru hodinových ručiček, který se nachází v jiném racku. 
\end{itemize}

\subsection{Donašeči (Snitches)}
Donašeči informují Cassandru o síťové topologii clusteru, tedy aby požadavky byly směřovány efektivně a aby repliky být uloženy do různých racků či datacenter. 

Na výběr máme několik implementací od jednoduchých, řešící topologii sítě maskováním ip adresy, až po komplexní, kde definujeme topologii ručně. Cassandra nabízí implementace i pro známé cloudové řešení od amazonu, které dokáže rozeznávat i jednotlivé regiony tohoto poskytovatele.

\subsubsection{Dynamické donášení}
Všechny implementace mají implicitně zapnutou tuto volbu, která se snaží omezit čtení z aktuálně přetížených uzlů, směřovat požadavky jinam a chytře tak využít celý cluster. 

\subsection{Úrovně konzistence}
Cassandra nabízí různé úrovně konzistence pro čtení i zápis. Možné hodnoty a jejich vlastnosti jsou popsané v tabulce \ref{casareads} a tabulce \ref{casawrites}

\begin{table}
  %  \begin{tabular}{|l|l|}
\begin{tabularx}{\textwidth}{ |l|X| }

    \hline
    Úroveň       & Popis                                                                                                                                                                                                                                                                                  \\ \hline
    ANY          & Zápis se musí provézt alespon na jeden uzel. Pokud jsou všechny uzly, kam má být replika umístěna nedostupné, můžeme provézt zápis do odkladového úložiště. Tyto data však nebudou přístupná ke čtení do té doby, než alespoň jeden uzel vlastnící repliku těchto dat nebude dostupný. \\ \hline
    ONE          & Data musí být zapsána do commit logu a paměti alespoň jednoho uzlu, kde má být replika umístěna                                                                                                                                                                                        \\ \hline
    TWO          & Data musí být zapsána do commit logu a paměti alespoň dvou uzlů, kde má být replika umístěna                                                                                                                                                                                           \\ \hline
    THREE        & Data musí být zapsána do commit logu a paměti alespoň tří uzlů, kde má být replika umístěna                                                                                                                                                                                            \\ \hline
    QUORUM       & Data musí být zapsána do commit logu a paměti na větší polovině uzlů, kde má být replika umístěna                                                                                                                                                                                      \\ \hline
    LOCAL\_QUORUM & Data musí být zapsána do commit logu a paměti na větší polovině uzlů kde má být replika umístěna v datacentru koordinátorského uzlu,                                                                                                                                                   \\ \hline
    EACH\_QUORUM & Data musí být zapsána do commit logu a paměti na větší polovině uzlů kde má být replika umístěna v každém datacentru,                                                                                                                                                                  \\ \hline
    ALL          & Data musí být zapsána do commit logu a paměti na všech uzlech kde má být replika umístěna.                                                                                                                                                                                             \\ \hline
    \end{tabularx}
    \caption {Úrovně konzistence pro zápis}
\label{casawrites}

\end{table}


\begin{table}[!h]
%    \begin{tabular}{|l|l|}
\begin{tabularx}{\textwidth}{ |l|X| }

    \hline
    Úroveň       & Popis                                                                                                                    \\ \hline
    ONE          & Vrátí odpověd z nejbližšího uzlu a na pozadí spustí opravné čtení                                                        \\ \hline
    TWO          & Vrátí aktuální data z dvou nejbližších uzlů (majících repliky)                                                           \\ \hline
    THREE        & Vrátí aktuální data z dvou nejbližších uzlů (majících repliky)                                                           \\ \hline
    QUORUM       & Vrátí nejnovější data potom co odpoví větší polovina uzlů majících repliky                                               \\ \hline
    LOCAL\_QUORUM & Vrátí nejnovější data potom co odpoví větší polovina uzlů majících repliky ze stejného datacentra v jakém je koordinátor \\ \hline
    EACH\_QUORUM & Vrátí nejnovější data potom co odpoví větší polovina uzlů majících repliky z každého datacentra                          \\ \hline
    ALL          & Vrátí nejnovější data potom co odpoví větší polovina všech uzlů majících repliky                                         \\ \hline
    \end{tabularx}
    \caption {Úrovně konzistence pro čtení}
\label{casareads}

\end{table}


\subsection{Eventuální konzistence}
Cassandra disponuje eventuální konzistencí. to znamená, že úroveň konzistence se mění. Cassandra tímto obchází zažité paradigma CAP theorému a implementuje tak Segmentovanou konzistenci, kterou jsme popisovali dříve. Je na architektovi a případně vývojáři kdy a jakou úroveň konzistence použije. Cassandra umožňuje zvolit  konzistenci pro každý požadavak zvlášť, dokonce i pokaždé jinou pro stejný dotaz. Toto dává architektům a vývojářům obrovskou výhodu při návrhu i samotné implementaci. 

\subsection{Výběr vhodné konzistence}
Univerzální pravidlo pro výběr vhodné úrovně neexistuje, záleží vždy na konkrétní situaci a akci. Obecně ovšem platí, že pokud potřebujeme data rychle volíme nížší úrovně, naopak pokud vyžadujeme co nejkatuálnější data, použíjeme nejvyšší úrovně. 

\section{Klientské dotazy}
Jak bylo zmíněno v předchozích sekcích, Cassandra je plně symetrický systém a tudíž požadavky (jak na čtení, tak zápis) mohou přicházet na kterýkoliv! uzel v clusteru. Uzel který požadavek přijme se stává koordinátorem a zajištuje, aby byl požadavek úspěšně dokončen. Koordinátor určuje, který uzel vlastní která data na základě Rozdělovací a replikační strategie. 

\subsection{Zápis}
Koordinátor pošle požadavek o zápis na \textbf{všechny} uzly, které podle  rozdělovací a replikační vlastni tyto data. Na tyto uzly budou data zapsána nehledě na úroveň konzistence. Konzistence nám určuje na kolik odpovědí od uzlů bude koordinátor čekat, než ohlasí požadavek za úspěšný. Úspěšný zápis znamená, že data byla zapsána do commit logu a do mezi paměti. 

Pokud tedy máme cluster o 12 uzlech a data zapisujeme s úrovní ONE a replikační faktor máme 3. Koordinátor (v tomto případě uzel 10) pošle požadavek o zápis na uzly 1,2 a 7. Koordinátor prohlasí požadavek za úspěšný, když mu přijde prvni odpověď od kteréhokoliv uzlu.

\begin{figure}[h]
\centering
\includegraphics[scale=0.5]{images/write}
\caption{Ukázka zápisu dat s replikačním faktorem 3 a konzistencí 1}
\label{fig:vnodes}
\end{figure}


\subsection{Čtení}
Čtení probíha dvěma možnými způsoby.

\begin{itemize}
\item Přímé čtení probíhá podobně jako zápis a podrobněji ho rozepíšeme
\item Čtení a opravení dat na pozadí, jedná se o složitější mechanismus, který popíšeme v následující sekci.
\end{itemize}

Přímé čtení tedy funguje podobně jako zápis. Koordinátor odešle požadavek na získání dat všem uzlům, které data vlastní a úroveň konzistence určuje na kolik správných odpovědí budeme čekat. Data v uzlech mohou být nekonzistentní cassandra považuje za správná ta nejnovější data. Konzistenční úroveň u čtení tedy říká na kolik uzlů mi musí vrátit aktuální data. 

V následujícím příkladu uvažujeme stejný cluster jako při zápisu a vyžadujeme data s konzistenční úorvní QUORUM a 2 ze 3 uzlů nám vrátili aktuální data a tedy považujeme požadavek za úspěšný a koordinátor vrátí klientovi data.


\begin{figure}[h]
\centering
\includegraphics[scale=0.55]{images/read}
\caption{Ukázka čtení dat s replikačním faktorem 3 a konzistencí Quorum}
\label{fig:vnodes}
\end{figure}


\section{Zachování konzistence}
Jak bylo naznačeno, při zápisu se může stát, že je uzel nedostupný a nedostane tedy data, která by měl zapsat a tím vznikají nekonzistence dat. Cassandra má 3 interní mechanismy jak si s nekonzistencí dat poradit. 

\begin{itemize}
\item \textbf{Read Repair} Pokud při čtení Cassandra zjistí, že má nějaký uzel nekonzistentní data, po dokončení požadavku pošle na pozadí aktualizační požadavek a data na těchto uzlech aktualizuje. Tato možnost je konfigurovatelná.
\item \textbf{Anti Entropy Node Repair} Pokud byl uzel dlouho nedostupný můžeme spustit opravný nástroj, který se sám dotáže na konzistenci svých dat ostatních uzlů a postupně je opraví. 
\item \textbf{Hinted Handoff} Pokud se stane, že je některý z uzlů nedostupný během zápisu, ostatní uzly s replikami si uloží takzvaný hint ohledně zápisu a ve chvíli, kdy se pomocí Gossip protokolu dozví, že daný uzel se vrátil zpět do clusteru, pošlou mu informaci o zápisu. Pokud použijeme při zápisu úroveň konzistence ANY, můžeme zápis provézt úspěšně i pokud budou všechny uzly, kterým data patří nedostupné. Data a hint se uloží na koordinátorském uzlu. Tato data ovšem nebudou dostupná ke čtení, dokud nebudou řádně zapsána, alespoň do jednoho z uzlů kam patří.
\end{itemize}

\section{Oprava dat na disku}
Přestože detailní struktura a ukládání dat na disku je nad rámec tohoto textu, je vhodné říct, že díky své struktuře zděděné od BigTable, neprovádí Cassandra žádné inplace úpravy ani mazání. Místo toho používá náhrobky a nové zápisy s aktuálnějšímy daty a časovým razítkem. Z toho plyne, že z hlediska zlepšení výkonu a ušetření místa na disku je potřeba čas od času provézt pročištění a sjednocení těchto append only souborů. Této technice se říká kompakce a jedná se o pokročilejší administrátorskou úlohu při práci s Cassandrou a tudíž mimo rámec tohoto textu. 

\section{Datový model}
Reprezentaci dat a ukládací engine si Cassandra vypujcila a přizpůsobila od BigTable. Cassandra je tedy spadá pod Key-Value storage systémy a WideColumn systémy. Těmto systémům se občas nadlehčeně říka \uv{Fancy hash table} což můžeme přeložit jako Honosná HashTabulka, toto označení se používá v souvislosti, že se jedná \uv{pouze} o ukládání klíče-hodnoty s pár funkcemi navíc. Vzhledem k vyspělosti těchto systému v dnešní době a rozsáhlosti funkcí, můžeme toto dehonestující označení považovat za pouhý historický komunitní slovní žert.

Cassandra v dnešní době disponuje moderním rozhraním, které si popíšeme v další sekci. Považuji ale za vhodné si připomenout takzvané \uv{legacy API}, které plně odkrývalo reprezentaci dat, která je důležitá pro pochopení toho, jak vlastně Cassandra funguje interně. V dnešní době je tento engine obalený přijemným SQL-Like rozhraním tyto znalosti se nám však mohou hodit při návrhu našeho doménového modelu a optimalizaci dotazů. V poslední řadě stále dnes můžeme použít toto staré Thrift API pro komunikaci s databází, jak si ale ukážeme není k tomu žádný rozumný důvod. 

\subsection{Změna myšlení}
Pro uživatelé přicházející ze světa relačních databází, byl velký problém se s úplně jiným přístupem vypořádat a neubránili se neustálému srovnávání přístupů a pojmů. Pro ovládnutí principů NoSQL a především efektivního a funkčního návrhu modelů v Cassandře je důležité myslět na 2 hlavní zásady. 

\begin{itemize}
\item \textbf{Denormalizace} Vše co jste se učili v SQL systémech o normalizaci dat zde zaprvé neplatí, ale zadruhé jen uškodí a vedou k nefunkčnímu návhrhu. Cassandra neumí JOIN příkazy které při normalizaci potřebujete a především NoSQL návrh vyžaduje vzhledem ke své nátuře úplně jiný přístup. 
\item \textbf{Redundance} Tento bod částečně souvisí s normalizací, která nám redundanci dat zakazuje a snaží se jí předejít. při návrhu NoSQL systémů se držíme paradigmatu, že redundance dat nevadí protože cena za lehke navýšení místa na disku je mnohem menší, než cena za čas a HW potřebný ke spuštění takových dotazů, které zvladnou získat potřebná data bez redundance. Ukládat tedy některá data dvakrát za účelem jejich následného snadného čtení je doporučený postup. 
\end{itemize}

Tento přístup byl pro mnoho vývojářů byl tězce adaptovatelný a i to byl jeden z důvodů, proč Cassandra používá CQL o kterém si povíme později. I přes všechny nadstavby je stále dobré vědět, jak Cassandra funguje interně a navrhovat tak kvalitní datové modely. 

\subsection{základní stavební kameny}
Přestože jsme se chtěli od relačního databázového světa odprostit, pro vysvětlení některých pojmů je nejjednodušší srovnání a proto v pár následujících bodech budu k relačním databázím přirovnávat.

\begin{itemize}
\item \textbf{Keyspace} - Udržuje pohromadě všechny Collumn family a replikační faktor, který se na CF přenáší. Každý Keyspace může mít jiný replikační faktor. Keyspace se dá chápat jako pojem databáze z relačního světa, má nějaké vlastnosti a ty plati pro všchny Collumn Families(tabulky), které sdružuje.
\item \textbf{Column Family} - Collumn Family sdružuje řádky, které obsahují sloupce v relačním světě bychom mohli CF přirovnat k tabulce, která udržuje jednotlivé záznamy. 
\item \textbf{Row} - Row je řádka identifikována jednoznačným klíčem (primárním) a obsahuje jednotlivé sloupce s daty, každá řádka může obsahovat sloupce s odlišnými daty, tedy řádky nemají pevnou strukturu. Řádky bychom mohli přirovnat k jednotlivým záznamům v tabulce relační databáze, ale s tím rozdílem, že řádky v Cassandře nemají pevný formát. 
\item \textbf{Column} - Sloupec je nejmenší a finální entitou udržující data v Cassandře. Sloupec má svůj název a hodnotu. Sloupce mají také časové razítko, které určuje čas vložení a tedy dle něj se koordinatoři rozhodují zda jsou sloupce z ruzných uzlů aktuální, či nikoliv. Cassandra disponuje těmito druhy sloupců

\begin{itemize}
\item \textbf{Standard} -  Standardní obyčejný sloupec, který uchovává jednu hodnotu
\item \textbf{Composite} - Spojený sloupec se používá pokud je primární klíč složený z více sloupců, poté názvy sloupců obsahují svůj původní název rozšířený o druhou část primárního klíče. 
\item \textbf{Expiring} - Sloupce s omezenou dobou platnosti se hodí pokud chceme životnost dat omezit nějakou dopředu známou dobou po kterou jsou data platná. Po vypršení této lhůty jsou data z databáze vymazána.
\item \textbf{Counter} - Čítací sloupce můžeme využít pokud chceme inkrementálně zvyšovat hodnotu v daném sloupci. Tato metoda se však příliš nepoužívá a raději se data předpočítávají průběžně. 
\end{itemize}
\end{itemize}



Z této architektury tedy lze naprvní pohled usoudit, že Cassandra je tedy opravdu pouze Key-Value storage systém a modelování komplikovanějších datových modelů je opravdu náročnou činností. Velkou výhodou je možnost pojmenování sloupce jakýmkoliv klíčem a tak si i do názvu můžeme zakódovat některé užitečné informace, jako nepříklad informaci o pořadí. K datům v této struktuře lze přistupovat pomocí Thrift API, které nám otevírá ukládací datový model Cassandry napřímo a data tak čteme v syrové podobě, tak jak jsou uchována přímo v databázi. Tento postup je však zastaralý a pro komunikaci s databází se doporučuje nástupce tohoto rozhraní a tedy dotazovací jazyk CQL. 

\begin{figure}[!h]
\centering
\includegraphics[scale=0.95]{images/static_column_family}
\caption{Schéma datového modelu v Column Family}
\label{fig:vnodes}
\end{figure}

\begin{figure}[!h]
\centering
\includegraphics[scale=1.5]{images/column}
\caption{datového modelu Column}
\label{fig:vnodes}
\end{figure}

\section{CQL}
Přestože bylo v předchozí sekci řečeno, že Thrift je zastaralý a CQL je doporučená cesta, jak se dotazovat nad daty, datový model, je stále hodně důležitý a je potřebu mu rozumět, abychom dokázali moel navrhnout funkčně a efektivně. Thrift nám tento model přímo odhaluje a CQL je abstrakce, která nám dovoluje pracovat s daty obdobně jako jsme zvyklí v relačním světě s tím rozdílem, že si poté dotazy převede na dotazy nad zmíněným interním datovým modelem. 

CQL je dotazovací jazyk podobny SQL, přes který lze vytvářet datové struktury, tabulky, které mají stejný formát jako tabulky relační, avšak na pozadí se struktury i dotazy nad nimi převedou na datový model popsaný v předchozí kapitole. CQL neumožňuje žádný příkaz z \uv{rodiny} JOIN. Přestože by se mohlo zdát, že pevná struktura nás omezuje oproti předchozím návrhům, opak je pravdou pomocí CQL jsme schopni vymodelovat data o různých hodnotách. Konkrétní příklady budou popsány v následující kapitole. 

\newpage

\begin{lstlisting}[caption={Tvorba jednoduché tabulky pomocí CQL},label=CQL1]
CREATE TABLE songs (
  id uuid PRIMARY KEY,
  title text,
  album text,
  artist text,
  data blob
 );
\end{lstlisting}

jak je z kódu \ref{CQL1} patrné, kód se opravdu mnoho neliší. Jazyk má určitá omezení a popsání všech omezení a funkcí jazyka by vydalo na samostatnou práci. Krátce bych chtěl jen upozornit, že jazyk podporuje indexy na sloupcích a také datové kolekce. Nejdůležitější funkce a omezení budou rozebrány v rámci případů užití v následující kapitole. Na závěr přikládám tabulku převodů interních typů na typy využívané v jazyce CQL.

\begin{table}[h]

    \begin{tabular}{|l|l|l|}
    \hline
    Internal Type     & CQL Name      & Description                                    \\ \hline
    BytesType         & blob          & Arbitrary hexadecimal bytes (no validation)    \\ \hline
    AsciiType         & ascii         & US-ASCII character string                      \\ \hline
    UTF8Type          & text, varchar & UTF-8 encoded string                           \\ \hline
    IntegerType       & varint        & Arbitrary-precision integer                    \\ \hline
    Int32Type         & int           & 4-byte integer                                 \\ \hline
    LongType          & bigint        & 8-byte long                                    \\ \hline
    UUIDType          & uuid          & Type 1 or type 4 UUID                          \\ \hline
    TimeUUIDType      & timeuuid      & Type 1 UUID only (CQL3)                        \\ \hline
    DateType          & timestamp     & Date plus time, encoded as 8 bytes since epoch \\ \hline
    BooleanType       & boolean       & true or false                                  \\ \hline
    FloatType         & float         & 4-byte floating point                          \\ \hline
    DoubleType        & double        & 8-byte floating point                          \\ \hline
    DecimalType       & decimal       & Variable-precision decimal                     \\ \hline
    CounterColumnType & counter       & Distributed counter value (8-byte long)        \\ \hline
    \end{tabular}
    \caption {Interní Datové typy}
   \label{datatypes}
\end{table}


\chapter{Vhodné případy užití}
\section{Analýza Access logů}
\section{Image Servery}
\section{Sběr a vyhodnocení transakčních událostí}


\chapter{Implementace některých případu užití}
\section{Testovací prostředí}
\section{Implementace Access Log analyzéru}

\begin{conclusion}
	%sem napište závěr Vaší práce
\end{conclusion}

\nocite{*}
\bibliographystyle{csn690}
\bibliography{mybibliographyfile}

\appendix

\chapter{Seznam použitých zkratek}
% \printglossaries
\begin{description}
	\item[GUI] Graphical user interface
	\item[XML] Extensible markup language
\end{description}

\chapter{Obsah přiloženého CD}

%upravte podle skutecnosti

\begin{figure}
	\dirtree{%
		.1 readme.txt\DTcomment{stručný popis obsahu CD}.
		.1 exe\DTcomment{adresář se spustitelnou formou implementace}.
		.1 src.
		.2 impl\DTcomment{zdrojové kódy implementace}.
		.2 thesis\DTcomment{zdrojová forma práce ve formátu \LaTeX{}}.
		.1 text\DTcomment{text práce}.
		.2 thesis.pdf\DTcomment{text práce ve formátu PDF}.
		.2 thesis.ps\DTcomment{text práce ve formátu PS}.
	}
\end{figure}

\end{document}
